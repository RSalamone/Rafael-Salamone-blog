\documentclass[]{article}
\usepackage{lmodern}
\usepackage{amssymb,amsmath}
\usepackage{ifxetex,ifluatex}
\usepackage{fixltx2e} % provides \textsubscript
\ifnum 0\ifxetex 1\fi\ifluatex 1\fi=0 % if pdftex
  \usepackage[T1]{fontenc}
  \usepackage[utf8]{inputenc}
\else % if luatex or xelatex
  \ifxetex
    \usepackage{mathspec}
  \else
    \usepackage{fontspec}
  \fi
  \defaultfontfeatures{Ligatures=TeX,Scale=MatchLowercase}
\fi
% use upquote if available, for straight quotes in verbatim environments
\IfFileExists{upquote.sty}{\usepackage{upquote}}{}
% use microtype if available
\IfFileExists{microtype.sty}{%
\usepackage{microtype}
\UseMicrotypeSet[protrusion]{basicmath} % disable protrusion for tt fonts
}{}
\usepackage[margin=1in]{geometry}
\usepackage{hyperref}
\hypersetup{unicode=true,
            pdftitle={Article 1: Koala epidemic provides lesson in how DNA protects itself from viruses},
            pdfborder={0 0 0},
            breaklinks=true}
\urlstyle{same}  % don't use monospace font for urls
\usepackage{longtable,booktabs}
\usepackage{graphicx,grffile}
\makeatletter
\def\maxwidth{\ifdim\Gin@nat@width>\linewidth\linewidth\else\Gin@nat@width\fi}
\def\maxheight{\ifdim\Gin@nat@height>\textheight\textheight\else\Gin@nat@height\fi}
\makeatother
% Scale images if necessary, so that they will not overflow the page
% margins by default, and it is still possible to overwrite the defaults
% using explicit options in \includegraphics[width, height, ...]{}
\setkeys{Gin}{width=\maxwidth,height=\maxheight,keepaspectratio}
\IfFileExists{parskip.sty}{%
\usepackage{parskip}
}{% else
\setlength{\parindent}{0pt}
\setlength{\parskip}{6pt plus 2pt minus 1pt}
}
\setlength{\emergencystretch}{3em}  % prevent overfull lines
\providecommand{\tightlist}{%
  \setlength{\itemsep}{0pt}\setlength{\parskip}{0pt}}
\setcounter{secnumdepth}{0}
% Redefines (sub)paragraphs to behave more like sections
\ifx\paragraph\undefined\else
\let\oldparagraph\paragraph
\renewcommand{\paragraph}[1]{\oldparagraph{#1}\mbox{}}
\fi
\ifx\subparagraph\undefined\else
\let\oldsubparagraph\subparagraph
\renewcommand{\subparagraph}[1]{\oldsubparagraph{#1}\mbox{}}
\fi

%%% Use protect on footnotes to avoid problems with footnotes in titles
\let\rmarkdownfootnote\footnote%
\def\footnote{\protect\rmarkdownfootnote}

%%% Change title format to be more compact
\usepackage{titling}

% Create subtitle command for use in maketitle
\providecommand{\subtitle}[1]{
  \posttitle{
    \begin{center}\large#1\end{center}
    }
}

\setlength{\droptitle}{-2em}

  \title{Article 1: Koala epidemic provides lesson in how DNA protects itself
from viruses}
    \pretitle{\vspace{\droptitle}\centering\huge}
  \posttitle{\par}
    \author{true}
    \preauthor{\centering\large\emph}
  \postauthor{\par}
      \predate{\centering\large\emph}
  \postdate{\par}
    \date{10-16-2019}


\begin{document}
\maketitle

\hypertarget{article}{%
\section{Article}\label{article}}

\includegraphics{https://www.sciencedaily.com/images/2019/10/191010113231_1_540x360.jpg}
In animals, infections are fought by the immune system. Studies on an
unusual virus infecting wild koalas, by a team of researchers from the
University of Massachusetts Medical School and the University of
Queensland, reveal a new form of \textbf{``genome immunity.''} The study
appears October 10 in the journal Cell.

earRetroviruses, including pathogens like HIV, incorporate into the
chromosomes of host cells as part of their infectious lifecycle.
\textbf{Retroviruses} don't usually infect the \textbf{germ cells} that
produce sperm and eggs and are therefore usually not \textbf{passed}
from generation to generation, but this has happened several times
during evolution. Out of the entire 3 billion nucleotides of the human
genome, only 1.5\% of the sequence forms the 20,000 genes that code for
proteins -- and 8\% of the human genome comes from fragments of viruses.
These pathogen invasions of the genome have sometimes been beneficial.
For example, a gene \textbf{``co-opted''} from a virus is required for
formation of the placenta in all mammals, including humans.

Retroviral infection of germ cells has been a rare but important
\textbf{driving force} in human evolution. But how the germ cells in
mammals respond to pathogen invasion has not been previously described
and might be quite different than other cells of the body. KoRV-A is a
retrovirus \textbf{sweeping through} the wild koala population of
Australia and is associated with susceptibility to infection and cancer.
KoRV-A spreads between individual animals, like most viruses.
\textbf{Surprisingly}, KoRV-A also infects the \textbf{germline} cells,
and most wild koalas are born with this pathogen as part of the genetic
material of every cell in the body. The team used this system to see how
germ cells respond to a retrovirus. Their findings suggest that that
germ cells recognize an essential step in the viral life cycle and turn
it against the invader to \textbf{suppress} genome infection. These
studies \textbf{shed new light on} interactions between viruses and the
genetic \textbf{``blueprint,''} written in the genome.

``KoRV-A infection of the koala germline is happening now, and lets us
look at genome evolution in real time,'' says William E. Theurkauf, PhD,
professor of molecular medicine at UMass Medical School, a senior author
of the study.

``What we are seeing with koalas is something that every organism on the
planet has \textbf{gone through}. Animals get infected by retroviruses
that enter the germline cells. These viruses multiply and insert into
the chromosomes, altering host genome organization and function, and the
process continues until the invader is \textbf{tamed} by the host. At
the end of this infection cycle, the host has changed,'' says co-senior
author Zhiping Weng, professor in the Program of Bioinformatics and
\textbf{Integrative Biology} at UMass Medical School.

``What we've uncovered, we believe, is an''innate" genome immune system
that can tell a virus from one of your genes," says Theurkauf. ``We
think this is getting at how your genome says, `This is something we
want; this is a gene.' And, `That is something we don't want; that's a
virus'.''

Most host genes are interrupted by \textbf{spacer sequences} called
introns, which are removed in a process called \textbf{splicing}, to
produce functional mRNAs that can make proteins. Splicing is a
\textbf{hallmark} of normally cellular genes. Retroviruses also have
introns, which are removed to make a protein that forms the envelope
that \textbf{surrounds} the virus particle. However, these invaders also
have to produce an ``unspliced'' RNA, which is essential to replication
and infection. The appears to be critical, as germ cells recognize these
virus-specific RNAs and \textbf{chop} them into a distinct class of
small RNAs, called ``sense'' piRNAs, which block the formation of the
virus. Preliminary studies suggest that this process is conserved from
insects to mammals.

The team is working to expand their findings. ``First, we're trying to
\textbf{figure out} is how the virus got into the germline in the first
place,'' says Weng. She and Theurkauf will conduct addition experiments
to determine the \textbf{machinery} in the cells that recognize the
difference in viral RNA, and, finally, they hope to better understand
the process of \textbf{chopping up} the unspliced RNA transcripts, so
they are no longer functional.

``We think we can \textbf{sort that out} by looking at koalas,'' says
Theurkauf.

This work was supported by in part by the National Institutes of Health
and the Chinese National Natural Science Foundation.

Word count : 711 words

Source:
\url{https://www.sciencedaily.com/releases/2019/10/191010113231.htm}

\hypertarget{vocabulary}{%
\section{Vocabulary}\label{vocabulary}}

\begin{longtable}[]{@{}lll@{}}
\toprule
\begin{minipage}[b]{0.18\columnwidth}\raggedright
Word from the text\strut
\end{minipage} & \begin{minipage}[b]{0.18\columnwidth}\raggedright
Synonym/definition in English\strut
\end{minipage} & \begin{minipage}[b]{0.56\columnwidth}\raggedright
French translation\strut
\end{minipage}\tabularnewline
\midrule
\endhead
\begin{minipage}[t]{0.18\columnwidth}\raggedright
Genome immunity\strut
\end{minipage} & \begin{minipage}[t]{0.18\columnwidth}\raggedright
resistance to disease with one's genome\strut
\end{minipage} & \begin{minipage}[t]{0.56\columnwidth}\raggedright
Immunité génomique\strut
\end{minipage}\tabularnewline
\begin{minipage}[t]{0.18\columnwidth}\raggedright
Retroviruses\strut
\end{minipage} & \begin{minipage}[t]{0.18\columnwidth}\raggedright
One of a group of viruses that cause cancers and include the famou AIDS
virus)\strut
\end{minipage} & \begin{minipage}[t]{0.56\columnwidth}\raggedright
Rétrovirus\strut
\end{minipage}\tabularnewline
\begin{minipage}[t]{0.18\columnwidth}\raggedright
Germ cells\strut
\end{minipage} & \begin{minipage}[t]{0.18\columnwidth}\raggedright
sexual reproductive cell\strut
\end{minipage} & \begin{minipage}[t]{0.56\columnwidth}\raggedright
Cellules germinatives\strut
\end{minipage}\tabularnewline
\begin{minipage}[t]{0.18\columnwidth}\raggedright
To pass\strut
\end{minipage} & \begin{minipage}[t]{0.18\columnwidth}\raggedright
to transfer to\strut
\end{minipage} & \begin{minipage}[t]{0.56\columnwidth}\raggedright
passer/transférer à\strut
\end{minipage}\tabularnewline
\begin{minipage}[t]{0.18\columnwidth}\raggedright
Co-opted\strut
\end{minipage} & \begin{minipage}[t]{0.18\columnwidth}\raggedright
appropriated\strut
\end{minipage} & \begin{minipage}[t]{0.56\columnwidth}\raggedright
Co-opté/approprié\strut
\end{minipage}\tabularnewline
\begin{minipage}[t]{0.18\columnwidth}\raggedright
Driving force\strut
\end{minipage} & \begin{minipage}[t]{0.18\columnwidth}\raggedright
something that plays a key role\strut
\end{minipage} & \begin{minipage}[t]{0.56\columnwidth}\raggedright
Force motrice\strut
\end{minipage}\tabularnewline
\begin{minipage}[t]{0.18\columnwidth}\raggedright
To sweep through\strut
\end{minipage} & \begin{minipage}[t]{0.18\columnwidth}\raggedright
To spread quickly\strut
\end{minipage} & \begin{minipage}[t]{0.56\columnwidth}\raggedright
Se répandre dans\strut
\end{minipage}\tabularnewline
\begin{minipage}[t]{0.18\columnwidth}\raggedright
Suprisingly\strut
\end{minipage} & \begin{minipage}[t]{0.18\columnwidth}\raggedright
Unexpectedly\strut
\end{minipage} & \begin{minipage}[t]{0.56\columnwidth}\raggedright
Etonnamment\strut
\end{minipage}\tabularnewline
\begin{minipage}[t]{0.18\columnwidth}\raggedright
Germline\strut
\end{minipage} & \begin{minipage}[t]{0.18\columnwidth}\raggedright
The germ line is the set of cells from stem cells to gametes\strut
\end{minipage} & \begin{minipage}[t]{0.56\columnwidth}\raggedright
Lignée germinale\strut
\end{minipage}\tabularnewline
\begin{minipage}[t]{0.18\columnwidth}\raggedright
To suppress\strut
\end{minipage} & \begin{minipage}[t]{0.18\columnwidth}\raggedright
To eliminate\strut
\end{minipage} & \begin{minipage}[t]{0.56\columnwidth}\raggedright
éliminer\strut
\end{minipage}\tabularnewline
\begin{minipage}[t]{0.18\columnwidth}\raggedright
To shed new light on\strut
\end{minipage} & \begin{minipage}[t]{0.18\columnwidth}\raggedright
To clarify/explain\strut
\end{minipage} & \begin{minipage}[t]{0.56\columnwidth}\raggedright
apporter un nouvel éclairage à\strut
\end{minipage}\tabularnewline
\begin{minipage}[t]{0.18\columnwidth}\raggedright
Blueprint\strut
\end{minipage} & \begin{minipage}[t]{0.18\columnwidth}\raggedright
Plan\strut
\end{minipage} & \begin{minipage}[t]{0.56\columnwidth}\raggedright
Plan\strut
\end{minipage}\tabularnewline
\begin{minipage}[t]{0.18\columnwidth}\raggedright
To go through\strut
\end{minipage} & \begin{minipage}[t]{0.18\columnwidth}\raggedright
To endure\strut
\end{minipage} & \begin{minipage}[t]{0.56\columnwidth}\raggedright
Traverser\strut
\end{minipage}\tabularnewline
\begin{minipage}[t]{0.18\columnwidth}\raggedright
Tamed\strut
\end{minipage} & \begin{minipage}[t]{0.18\columnwidth}\raggedright
Domesticated\strut
\end{minipage} & \begin{minipage}[t]{0.56\columnwidth}\raggedright
Apprivoisé\strut
\end{minipage}\tabularnewline
\begin{minipage}[t]{0.18\columnwidth}\raggedright
Integrative Biology\strut
\end{minipage} & \begin{minipage}[t]{0.18\columnwidth}\raggedright
a recent field of biology that studies living organisms as the systems
they are in reality, as opposed to historical approaches that tend to
break down the study at all levels, in biology, physiology,
biochemistry\strut
\end{minipage} & \begin{minipage}[t]{0.56\columnwidth}\raggedright
Biology intégrative\strut
\end{minipage}\tabularnewline
\begin{minipage}[t]{0.18\columnwidth}\raggedright
Spacer sequences\strut
\end{minipage} & \begin{minipage}[t]{0.18\columnwidth}\raggedright
sequences that separate other sequences from each other\strut
\end{minipage} & \begin{minipage}[t]{0.56\columnwidth}\raggedright
Séquences d'espacement\strut
\end{minipage}\tabularnewline
\begin{minipage}[t]{0.18\columnwidth}\raggedright
Splicing\strut
\end{minipage} & \begin{minipage}[t]{0.18\columnwidth}\raggedright
A process that removes the intervening, non-coding sequences of genes
(introns) from pre-mRNA and joins the protein-coding sequences (exons)
together in order to enable translation of mRNA into a protein\strut
\end{minipage} & \begin{minipage}[t]{0.56\columnwidth}\raggedright
Epissage\strut
\end{minipage}\tabularnewline
\begin{minipage}[t]{0.18\columnwidth}\raggedright
Hallmark\strut
\end{minipage} & \begin{minipage}[t]{0.18\columnwidth}\raggedright
Distinctive characteristic\strut
\end{minipage} & \begin{minipage}[t]{0.56\columnwidth}\raggedright
Caractéristique principale\strut
\end{minipage}\tabularnewline
\begin{minipage}[t]{0.18\columnwidth}\raggedright
To surround\strut
\end{minipage} & \begin{minipage}[t]{0.18\columnwidth}\raggedright
To be all arround\strut
\end{minipage} & \begin{minipage}[t]{0.56\columnwidth}\raggedright
Entourer\strut
\end{minipage}\tabularnewline
\begin{minipage}[t]{0.18\columnwidth}\raggedright
To chop\strut
\end{minipage} & \begin{minipage}[t]{0.18\columnwidth}\raggedright
Cut into pieces\strut
\end{minipage} & \begin{minipage}[t]{0.56\columnwidth}\raggedright
Couper\strut
\end{minipage}\tabularnewline
\begin{minipage}[t]{0.18\columnwidth}\raggedright
To figure out\strut
\end{minipage} & \begin{minipage}[t]{0.18\columnwidth}\raggedright
To understand\strut
\end{minipage} & \begin{minipage}[t]{0.56\columnwidth}\raggedright
Comprendre\strut
\end{minipage}\tabularnewline
\begin{minipage}[t]{0.18\columnwidth}\raggedright
Machinery\strut
\end{minipage} & \begin{minipage}[t]{0.18\columnwidth}\raggedright
The organization or structure of something or for doing something\strut
\end{minipage} & \begin{minipage}[t]{0.56\columnwidth}\raggedright
Machinerie\strut
\end{minipage}\tabularnewline
\begin{minipage}[t]{0.18\columnwidth}\raggedright
Process of chopping up\strut
\end{minipage} & \begin{minipage}[t]{0.18\columnwidth}\raggedright
Cutting processus\strut
\end{minipage} & \begin{minipage}[t]{0.56\columnwidth}\raggedright
Processus de découpage\strut
\end{minipage}\tabularnewline
\begin{minipage}[t]{0.18\columnwidth}\raggedright
To sort out\strut
\end{minipage} & \begin{minipage}[t]{0.18\columnwidth}\raggedright
To resolve\strut
\end{minipage} & \begin{minipage}[t]{0.56\columnwidth}\raggedright
Régler\strut
\end{minipage}\tabularnewline
\bottomrule
\end{longtable}

\hypertarget{analysis-table}{%
\section{Analysis table}\label{analysis-table}}

\begin{longtable}[]{@{}ll@{}}
\toprule
\begin{minipage}[b]{0.47\columnwidth}\raggedright
\_\_\_\_\_\_\_\_\_\_\_\_\_\strut
\end{minipage} & \begin{minipage}[b]{0.47\columnwidth}\raggedright
\_\_\_\_\_\_\_\_\_\_\_\_\strut
\end{minipage}\tabularnewline
\midrule
\endhead
\begin{minipage}[t]{0.47\columnwidth}\raggedright
Researchers\strut
\end{minipage} & \begin{minipage}[t]{0.47\columnwidth}\raggedright
University of Massachusetts Medical School and the University of
Queensland (lead authors: Weng. She and Theurkauf)\strut
\end{minipage}\tabularnewline
\begin{minipage}[t]{0.47\columnwidth}\raggedright
Published in ?\strut
\end{minipage} & \begin{minipage}[t]{0.47\columnwidth}\raggedright
ScienceDaily, October 10, 2019\strut
\end{minipage}\tabularnewline
\begin{minipage}[t]{0.47\columnwidth}\raggedright
General topic ?\strut
\end{minipage} & \begin{minipage}[t]{0.47\columnwidth}\raggedright
How koala epidemic can help us to understand the way koala succeeded to
protect itself from viruses.\strut
\end{minipage}\tabularnewline
\begin{minipage}[t]{0.47\columnwidth}\raggedright
Procedure/what was examined?\strut
\end{minipage} & \begin{minipage}[t]{0.47\columnwidth}\raggedright
They examined how KoRV-A, a retrovirus which sweeps through the wild
population, infects germ cells of koalas.\strut
\end{minipage}\tabularnewline
\begin{minipage}[t]{0.47\columnwidth}\raggedright
Conclusions/discovery ?\strut
\end{minipage} & \begin{minipage}[t]{0.47\columnwidth}\raggedright
Koalas develop an innate genome immune system that can distinguish a
virus from one of their genes. Germ cells recognize an indispensable
step in the viral life cycle and turn it against the intruder to
eliminate genome infection.\strut
\end{minipage}\tabularnewline
\begin{minipage}[t]{0.47\columnwidth}\raggedright
Remaining questions\strut
\end{minipage} & \begin{minipage}[t]{0.47\columnwidth}\raggedright
How the virus got into the germline ? A better understanding of the
process of cutting the unspliced RNA transcripts in order to they are no
longer functional ?\strut
\end{minipage}\tabularnewline
\bottomrule
\end{longtable}


\end{document}
